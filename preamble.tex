%! TEX root = main.tex

\usepackage[top=25truemm,bottom=25truemm,left=20truemm,right=20truemm]{geometry}
\usepackage{lipsum}
\usepackage{changepage}
\usepackage{setspace}
\usepackage[dvipdfmx]{graphicx}
\usepackage{here}
\usepackage{booktabs}
\usepackage{amsmath, amssymb, mathtools}
\usepackage{bm}
\usepackage{url}
\usepackage{multirow}


% 全角数字を定義
\makeatletter
\def\@arabicz#1{%
  \ifcase#1 0\or 1\or 2\or 3\or 4\or 5\or 6\or 7\or 8\or 9\or 10\or
  11\or 12\or 13\or 14\or 15\or 16\or 17\or 18\or 19\or 20\or
  21\or 22\or 23\or 24\or 25\or 26\else\@ctrerr\fi}
\def\arabicz#1{\expandafter\@arabicz\csname c@#1\endcsname}%123 全角
\makeatother


% セクション系
  % 書式を変更
  \renewcommand{\thesection}{\normalsize\arabicz{section}.}  % sectionを 「字.」 の形式に変更
  \renewcommand{\thesubsection}{\normalsize\arabicz{section}・\arabicz{subsection}}  % subsection を 「数字・数字」の形式に変更
  \renewcommand{\thesubsubsection}{\normalsize\arabicz{section}・\arabicz{subsection}・\arabicz{subsubsection}}  % subsection を 「数字・数字・数字」の形式に変更

  % 余白を調整
  \usepackage{titlesec}
  % zw 現在有効な全角漢字の幅
  % zh 現在有効な全角漢字の高さ
  % \titlespacing{セクションタイプ}{左の余白(インデント)}{上の余白}{下(セクションと本文の間)の余白} で設定する
  \titlespacing{\subsection}{1zw}{1zh}{0pt}
  \titlespacing{\subsubsection}{1zw}{1zh}{0pt}


% 参考文献リストのリスト番号形式を 1) 等の右小カッコのみに変更
\makeatletter
\def\@biblabel#1{#1)}
\makeatother


% 本文中のciteを上付き1)に変更
\usepackage{overcite}        % \usepackage[superscript]{cite} と同じ.
\makeatletter
\def\@citess#1{\mbox{$\m@th^{\hbox{\OverciteFont{#1)}}}$}}
\makeatother


% キャプション系
\usepackage[style=base]{caption}
  % 図のキャプション
  \captionsetup[figure]{
    font=bf,  % フォントをゴシックに,
    labelsep=quad,  % 図番号とキャプションの間の余白を設定
    belowskip=-1zh,  % キャプションとその下の文とかの間隔を0に(おそらく初期設定で+1zhになっているので-1zhすることで相殺)
    aboveskip=0pt  % 図とキャプションの間隔を0に
  }

  % 表のキャプション
  \captionsetup[table]{
    font=bf,  % フォントをゴシックに,
    labelsep=quad  % 図番号とキャプションの間の余白を設定
  }


% その他の余白系
  % 本文中に入れる(場所指定が[H]の場合の)図と表の上下の余白を1行分確保する
  \setlength\intextsep{\baselineskip}

  % 場所指定が[t]や[b]の場合の図や表同士の余白を1行分確保する
  \setlength\floatsep{\baselineskip}

  % 場所指定が[t]や[b]の場合の図や表と本文との余白を1行分確保する
  \setlength\textfloatsep{\baselineskip}

  % 2列の最後の余白を揃える(他の行間がバグる元なので可能な限り使わないようにしたい)
  \usepackage{flushend}
