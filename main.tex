\documentclass[10pt, a4j, uplatex, twocolumn]{jsarticle}

%! TEX root = main.tex

\usepackage[top=25truemm,bottom=25truemm,left=20truemm,right=20truemm]{geometry}
\usepackage{lipsum}
\usepackage{changepage}
\usepackage{setspace}
\usepackage[dvipdfmx]{graphicx}
\usepackage{here}
\usepackage{booktabs}
\usepackage{amsmath, amssymb, mathtools}
\usepackage{bm}
\usepackage{url}
\usepackage{multirow}

%! TEX root = main.tex

%%% 年報スタイルの設定

%% スタイル設定

% 全角数字を定義
\makeatletter
\def\@arabicz#1{%
  \ifcase#1 0\or 1\or 2\or 3\or 4\or 5\or 6\or 7\or 8\or 9\or 10\or
  11\or 12\or 13\or 14\or 15\or 16\or 17\or 18\or 19\or 20\or
  21\or 22\or 23\or 24\or 25\or 26\else\@ctrerr\fi}
\def\arabicz#1{\expandafter\@arabicz\csname c@#1\endcsname}%123 全角
\makeatother


% セクション系
  % 書式を変更
  \renewcommand{\thesection}{\normalsize\arabicz{section}.}  % sectionを 「字.」 の形式に変更
  \renewcommand{\thesubsection}{\normalsize\arabicz{section}・\arabicz{subsection}}  % subsection を 「数字・数字」の形式に変更
  \renewcommand{\thesubsubsection}{\normalsize\arabicz{section}・\arabicz{subsection}・\arabicz{subsubsection}}  % subsection を 「数字・数字・数字」の形式に変更

  % 余白を調整
  \usepackage{titlesec}
  % zw 現在有効な全角漢字の幅
  % zh 現在有効な全角漢字の高さ
  % \titlespacing{セクションタイプ}{左の余白(インデント)}{上の余白}{下(セクションと本文の間)の余白} で設定する
  \titlespacing{\section}{0pt}{1zh}{0pt}
  \titlespacing{\subsection}{1zw}{1zh}{0pt}
  \titlespacing{\subsubsection}{1zw}{0pt}{0pt}


% 参考文献リストのリスト番号形式を 1) 等の右小カッコのみに変更
\makeatletter
\def\@biblabel#1{#1)}
\makeatother


% 本文中のciteを上付き1)に変更
\usepackage{overcite}        % \usepackage[superscript]{cite} と同じ.
\makeatletter
\def\@citess#1{\mbox{$\m@th^{\hbox{\OverciteFont{#1)}}}$}}
\makeatother


% キャプション系
\usepackage[style=base]{caption}
  % 図のキャプション
  \captionsetup[figure]{
    font=bf,  % フォントをゴシックに,
    labelsep=quad,  % 図番号とキャプションの間の余白を設定
    belowskip=-1zh,  % キャプションとその下の文とかの間隔を0に(おそらく初期設定で+1zhになっているので-1zhすることで相殺)
    aboveskip=0pt  % 図とキャプションの間隔を0に
  }

  % 表のキャプション
  \captionsetup[table]{
    font=bf,  % フォントをゴシックに,
    labelsep=quad  % 図番号とキャプションの間の余白を設定
  }


% その他の余白系
  % 本文中に入れる(場所指定が[H]の場合の)図と表の上下の余白を1行分確保する
  \setlength\intextsep{\baselineskip}

  % 場所指定が[t]や[b]の場合の図や表同士の余白を1行分確保する
  \setlength\floatsep{\baselineskip}

  % 場所指定が[t]や[b]の場合の図や表と本文との余白を1行分確保する
  \setlength\textfloatsep{\baselineskip}

  % 2列の最後の余白を揃える(他の行間がバグる元なので可能な限り使わないようにしたい)
  \usepackage{flushend}

  % 2段組みの間にある余白の調整
  \setlength{\columnsep}{20mm}



%% タイトル設定(マクロでゴリ押し)

% 16pt用フォントサイズを定義
% \newcommand{\myFontLarge}[1]{{\fontsize{16pt}{0pt}\selectfont #1}}
\newcommand{\myFontLarge}[1]{{\fontsize{17.28pt}{0pt}\selectfont #1}}
% 標準フォントサイズは 10pt
% \large: 12pt
% \myFontLarge: 16pt -> 17.28pt (latexの仕様上16ptは無理)
% 出典: http://www.latex-cmd.com/style/size.html

% 要素のマクロを定義
\newcommand{\titleJP}[1]{
    \begin{spacing}{1.5}
        \gtfamily\mdseries\upshape \myFontLarge{#1}
    \end{spacing}
}  % 日本語タイトル
\newcommand{\authorJP}[1]{{\normalsize #1}\footnotemark[1]}  % 日本語著者
\newcommand{\titleEN}[1]{{\large #1}}  % 英語タイトル
\newcommand{\authorEN}[1]{{\normalsize #1}}  % 英語著者
\newcommand{\customAbstract}[1]{
    {\bf ABSTRACT} \\
    \vspace{\baselineskip}
    \begin{adjustwidth}{10mm}{10mm}
        \hspace{5ex} #1
    \end{adjustwidth}
}
\newcommand{\customKeywords}[1]{
    \begin{adjustwidth}{10mm}{10mm}
        {\bf KEY WORDS}: #1
    \end{adjustwidth}
}
% 出典: https://w.atwiki.jp/chaos987/pages/23.html

% タイトルとアブストラクト全体
\newcommand{\titleAndAbstractAndKeywords}[7]{
    \twocolumn[  % これで囲むと 1段組みにできる
    \begin{center}
        \vspace{3\baselineskip}  % 3行開ける
        \titleJP{#1}  % 日本語タイトル
        \vspace{\baselineskip}  % 1行(10pt)開ける
        \authorJP{#2} \\  % 日本語著者とフットノート所属
        \vspace{\baselineskip}
        \titleEN{#4} \\  % 英語タイトル
        \vspace{\baselineskip}
        \authorEN{#5} \\  % 英語著者
        \vspace{\baselineskip}
        \customAbstract{#6}  % 英語アブストラクト
    \end{center}
    \vspace{\baselineskip}
    \customKeywords{#7}
    \vspace{3\baselineskip}
    ]
    \footnotetext[1]{#3}  % \twocolumn[]の中だとfootnoteが使えないので処置

}


\begin{document}
%! TEX root = ../main.tex

% タイトルをゴリ押しで作成

%%% マクロゴリ押し(編集不要)----------------------------------------------

% 16pt用フォントサイズを定義
\newcommand{\myFontLarge}[1]{{\fontsize{16pt}{0pt}\selectfont #1}}
% 標準フォントサイズは 10pt
% \large: 12pt 
% \myFontLarge: 16pt
% 出典: http://www.latex-cmd.com/style/size.html

% 要素のマクロを定義
\newcommand{\titleJP}[1]{
        {\gtfamily\mdseries\upshape \myFontLarge{#1}}
}  % 日本語タイトル
% \newcommand{\titleJP}[1]{
%     \begin{spacing}[2.0]
%         #1
%     \end{spacing}
% }  % 日本語タイトル
\newcommand{\authorJP}[1]{{\normalsize #1}\footnotemark[1]}  % 日本語著者
\newcommand{\titleEN}[1]{{\large #1}}  % 英語タイトル
\newcommand{\authorEN}[1]{{\normalsize #1}}  % 英語著者
\newcommand{\customAbstract}[1]{
    {\bf ABSTRACT} \\
    \vspace{\baselineskip}
    \begin{adjustwidth}{10mm}{10mm}
        \hspace{5ex} #1
    \end{adjustwidth}
}
% 出典: https://w.atwiki.jp/chaos987/pages/23.html

% タイトルとアブストラクト全体
\newcommand{\titleAndAbstract}[6]{
    \twocolumn[  % これで囲むと 1段組みにできる
    \begin{center}
          \\  % 1行分の余白
        \vspace{\baselineskip}  % 1行(10pt)開ける
        \vspace{\baselineskip}  % 1行(10pt)開ける 上記と合わせて3行分の余白確保
        \titleJP{#1} \\  % 日本語タイトル
        \vspace{\baselineskip}
        \authorJP{#2} \\  % 日本語著者とフットノート所属
        \vspace{\baselineskip}
        \titleEN{#4} \\  % 英語タイトル
        \vspace{\baselineskip}
        \authorEN{#5} \\  % 英語著者
        \vspace{\baselineskip}
        \customAbstract{#6}  % 英語アブストラクト
    \end{center}
    ]
    \footnotetext[1]{#3}  % \twocolumn[]の中だとfootnoteが使えないので処置

}

%%% マクロゴリ押し(編集不要)ここまで ---------------------------------------

% タイトルとアブストラクトを入力
\titleAndAbstract
% 日本語タイトル
{真空蒸着した銅フタロシアニン薄膜の膜厚測定.真空蒸着した銅フタロシアニン薄膜の膜厚測定}
% 日本語著者と所属
{高専 太郎}{機械・電子システム工学専攻}
% 英語タイトル
{Evaluation of Thickness of Copper Phthalocyanine Films Formed \\ by Vacuum Deposition}
% 英語著者
{Taro KOSEN}
% 英語アブストラクト
{\lipsum[1]}


\section{セクション}
せくしょんの本文半角0,全角0.
日本語参考文献を参照する.自然言語処理を本気でやりたい人にはおすすめの本\cite{Natural_Language_Processing_by_Deep_Learning}です.
更に,英語の参考文献も参照する.単語を分散ベクトル表現にして,計算しようと試みたWord2Vecという考え方\cite{Mikolov_word2vec}があります.
\subsection{サブセクション}
ニューラルネットワークの誤差逆伝播法の中で,各エッジの重みを更新する手法の1つとして,
ADAM\cite{ADAM}がある.

\subsubsection{サブサブセクション}
サブサブセクションを書く.
日本語では章節項の項.

\section{画像の挿入}
画像を入れる場合は,画像の上部とキャプションの下に1行の空白ができるように配置する.
\begin{figure}[H]
    \centering
    \includegraphics[width=0.8\linewidth]{latex-logo.png}
    \caption{LaTeXのロゴ}
    \label{fig:latex}
\end{figure}
って感じに下にも余白を設ける.
\section{表の挿入}
\lipsum[1]
表を入れる場合も,表のキャプションの上と,表のしたに1行の空白ができるように配置する.
\begin{table}[H]
\centering
\caption{例の表}
\label{tab:example}
\begin{tabular}{@{}lll@{}}
\toprule
hoge & fuga & pyo \\
\midrule
11   & 12   & 13  \\
21   & 22   & 23 \\
\bottomrule
\end{tabular}
\end{table}
って感じに下にも余白を設ける.

数式式\eqref{eq:hoge}に示す.
\begin{equation}
    \frac{1}{2} + \frac{1}{3} = \frac{3 + 2}{6} = \frac{5}{6} \label{eq:hoge}
\end{equation}

本文の終わりを揃えるために導入した flushend.sty を使うと,
最後のページのあらゆる余白がバグる
(例えば上の式\eqref{eq:hoge}の上下の余白の大きさが違うとか)
傾向にあるので,
可能ならば使わずに手動で調整したほうが良いかもです.

\bibliography{mybib}
\bibliographystyle{junsrt}

\end{document}
