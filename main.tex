\documentclass[10pt, a4j, uplatex, twocolumn]{jsarticle}

%! TEX root = ../main.tex

\usepackage[top=25truemm,bottom=25truemm,left=20truemm,right=20truemm]{geometry}
\usepackage{lipsum}
\usepackage{changepage}
\usepackage{setspace}
\usepackage[dvipdfmx]{graphicx}
\usepackage{here}
\usepackage{booktabs}

% 全角数字を定義
\makeatletter
\def\@arabicz#1{%
  \ifcase#1 0\or 1\or 2\or 3\or 4\or 5\or 6\or 7\or 8\or 9\or 10\or
  11\or 12\or 13\or 14\or 15\or 16\or 17\or 18\or 19\or 20\or 
  21\or 22\or 23\or 24\or 25\or 26\else\@ctrerr\fi}
\def\arabicz#1{\expandafter\@arabicz\csname c@#1\endcsname}%123 全角
\makeatother

% セクションの書式を変更
\renewcommand{\thesection}{\normalsize\arabicz{section}.}
\renewcommand{\thesubsection}{\normalsize\arabicz{section}・\arabicz{subsection}}
\renewcommand{\thesubsubsection}{\normalsize\arabicz{section}・\arabicz{subsection}・\arabicz{subsubsection}}

% セクションの上の余白を調整
\usepackage{titlesec}
\titlespacing{\subsection}{1zw}{\baselineskip}{0pt}
\titlespacing{\subsubsection}{1zw}{\baselineskip}{0pt}

% 参考文献リストのリスト番号形式を 1) 等の右小カッコのみに変更
\makeatletter
\def\@biblabel#1{#1)}
\makeatother

% 本文中のciteを上付き1)に変更
\usepackage{overcite}        % \usepackage[superscript]{cite} と同じ.
\makeatletter
\def\@citess#1{\mbox{$\m@th^{\hbox{\OverciteFont{#1)}}}$}}
\makeatother

% キャプションのフォントをゴシックに,区切りを無しに
\usepackage{caption}
\captionsetup[figure]{font=bf, labelsep=quad}
\captionsetup[table]{font=bf, labelsep=quad}
\setlength\intextsep{\baselineskip}
\setlength\abovecaptionskip{0pt}

% 最後の余白を揃える
\usepackage{flushend}


\begin{document}
%! TEX root = main.tex

% タイトルをゴリ押しで作成

%%% マクロゴリ押し(編集不要)----------------------------------------------

% 16pt用フォントサイズを定義
% \newcommand{\myFontLarge}[1]{{\fontsize{16pt}{0pt}\selectfont #1}}
\newcommand{\myFontLarge}[1]{{\fontsize{17.28pt}{0pt}\selectfont #1}}
% 標準フォントサイズは 10pt
% \large: 12pt
% \myFontLarge: 16pt -> 17.28pt (latexの仕様上16ptは無理)
% 出典: http://www.latex-cmd.com/style/size.html

% 要素のマクロを定義
\newcommand{\titleJP}[1]{
    \begin{spacing}{1.5}
        \gtfamily\mdseries\upshape \myFontLarge{#1}
    \end{spacing}
}  % 日本語タイトル
\newcommand{\authorJP}[1]{{\normalsize #1}\footnotemark[1]}  % 日本語著者
\newcommand{\titleEN}[1]{{\large #1}}  % 英語タイトル
\newcommand{\authorEN}[1]{{\normalsize #1}}  % 英語著者
\newcommand{\customAbstract}[1]{
    {\bf ABSTRACT} \\
    \vspace{\baselineskip}
    \begin{adjustwidth}{10mm}{10mm}
        \hspace{5ex} #1
    \end{adjustwidth}
}
\newcommand{\customKeywords}[1]{
    \begin{adjustwidth}{10mm}{10mm}
        {\bf KEY WORDS}: #1
    \end{adjustwidth}
}
% 出典: https://w.atwiki.jp/chaos987/pages/23.html

% タイトルとアブストラクト全体
\newcommand{\titleAndAbstractAndKeywords}[7]{
    \twocolumn[  % これで囲むと 1段組みにできる
    \begin{center}
        \vspace{3\baselineskip}  % 3行開ける
        \titleJP{#1}  % 日本語タイトル
        \vspace{\baselineskip}  % 1行(10pt)開ける
        \authorJP{#2} \\  % 日本語著者とフットノート所属
        \vspace{\baselineskip}
        \titleEN{#4} \\  % 英語タイトル
        \vspace{\baselineskip}
        \authorEN{#5} \\  % 英語著者
        \vspace{\baselineskip}
        \customAbstract{#6}  % 英語アブストラクト
    \end{center}
    \vspace{\baselineskip}
    \customKeywords{#7}
    \vspace{3\baselineskip}
    ]
    \footnotetext[1]{#3}  % \twocolumn[]の中だとfootnoteが使えないので処置

}

%%% マクロゴリ押し(編集不要)ここまで ---------------------------------------


% タイトルとアブストラクトを入力
\titleAndAbstractAndKeywords
% 日本語タイトル
{真空蒸着した銅フタロシアニン薄膜の膜厚測定}
% 日本語著者と所属
{高専 太郎}{機械・電子システム工学専攻}
% 英語タイトル
{Evaluation of Thickness of Copper Phthalocyanine Films Formed \\ by Vacuum Deposition}
% 英語著者
{Taro KOSEN}
% 英語アブストラクト
{
Copper phthalocyanine (CuPc) is often used in organic light emitting devices.
We deposited CuPc films on silicon wafers by means of a vacuum deposition method.
We then tried to estimate film thickness using a Fourier transformed infrared spectrometer.
This method is applicable when the thickness is over several $\mu$ m.
The obtained thickness was compared with that measured with a surface profiler and thedifference between the two methods is under 44 \%.
}
% キーワード
{thin films, copper phthalocyanine}


\section{セクション番号ほげ}
せくしょんの本文半角0,全角0.
日本語参考文献を参照する.自然言語処理を本気でやりたい人にはおすすめの本\cite{Natural_Language_Processing_by_Deep_Learning}です.
更に,英語の参考文献も参照する.単語を分散ベクトル表現にして,計算しようと試みたWord2Vecという考え方\cite{Mikolov_word2vec}があります.
\subsection{サブセクション}
ニューラルネットワークの誤差逆伝播法の中で,各エッジの重みを更新する手法の1つとして,
ADAM\cite{ADAM}がある.

\subsubsection{サブ2乗セクション}
ほあsjどふぁじょsdんふぉあsdんゔぁ


\bibliography{mybib}
\bibliographystyle{junsrt}

\end{document}
